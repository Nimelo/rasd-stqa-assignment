\chapter{Models and Diagrams} \label{chp:models-and-diagrams}
	There is only one actor of a simulation tool shown in Figure \ref{fig:actors}. Idea of a simulation tool is to do a simulation of \gls{computing-platform}, which should provide information about accounting practices. In the simplest way this kind of behavior is shown in Figure \ref{fig:idea}. Action starts with a configuration of a simulation, next the chain of events inside simulation tool is performed giving the output of a simulation described in chapter \ref{s:system-features:configuration-feature}. The uses cases of a simulation tool are divided into two main boundary: simulation boundary and configuration boundary, which is presented in Figure \ref{fig:use-cases}. User can perform any action described in section \ref{s:system-features:communication-feature}. Use cases from the configuration boundary are not strictly connected with the software, although pretty well describes behavior of the configuration files. User can do four main actions using simulation software: configure tool, run simulation, generate template of configuration file or run simulation. All the internal actions are described in section in \ref{s:system-features:simulation-feature} or shown in Figure \ref{fig:scheduler}.