\chapter{Models and Diagrams} \label{chp:models-and-diagrams}
	There is only one actor of a simulation tool shown in Figure \ref{fig:actors}. Idea of a simulation tool is to do a simulation of \gls{computing-platform}, which should provide information about accounting practices. In the simplest way this kind of behavior is shown in Figure \ref{fig:idea}. Action starts with a configuration of a simulation, next the chain of events inside simulation tool is performed giving the output of a simulation described in chapter \ref{s:system-features:configuration-feature}. The uses cases of a simulation tool are divided into two main boundary: simulation boundary and configuration boundary, which is presented in Figure \ref{fig:use-cases}. User can perform any action described in section \ref{s:system-features:communication-feature}. Use cases from the configuration boundary are not strictly connected with the software, although pretty well describes behavior of the configuration files. User can do four main actions using simulation software: configure tool, run simulation, generate template of configuration file or run simulation. All the internal actions are described in section in \ref{s:system-features:simulation-feature} or shown in Figure \ref{fig:scheduler}.
	
	Idea of simulation mentioned before is a top layer data flow in a system presented a happy path in Figure \ref{fig:idea}. It is possible to distinguish three main processes with simulator: start of simulation, tracing progress and producing result.
	
	Scheduling algorithm presented in Figure \ref{fig:scheduler} has three steps. First step is to match a job to a different queue, which can be parametrized as described in section \ref{s:system-features:configuration-feature} or briefly presented in diagram itself. Next the validation is made about the time constraints -- queue must be in availability state in order to serve. If queue is serving the next check will be made, this time queue will try to allocate resources for the job. If there are enough resources the job will start their execution. In other case, similarly to case where the queue is not serving the job will be placed in a waiting queue.
	
	Use cases in a boundary starting with 'Configuration' relate to the section \ref{s:system-features:configuration-feature}. Each of those cases is visualization and counterpart of q requirement within mentioned before section.