\chapter{System Features} \label{chp:system-features}
	\begin{comment}
		$<$This template illustrates organizing the functional requirements for the 
		product by system features, the major services provided by the product. You may 
		prefer to organize this section by use case, mode of operation, user class, 
		object class, functional hierarchy, or combinations of these, whatever makes the 
		most logical sense for your product.$>$
	\end{comment}
	This chapter is divided into four main sections, which are responsible to contain list of requirements and explain processes within among those requirements. First section \ref{s:system-features:communication-feature} describes all the requirements connected with the user-interface, such us the responses for given action parameters. Next section \ref{s:system-features:configuration-feature} presents the way of configuring the system, containing very specified requirements. Simulation process is described in a \ref{s:system-features:simulation-feature} section. The last section \ref{s:system-features:reporting-feature} standardize the report format and the calculated values.
	
	\section{Communication Feature} \label{s:system-features:communication-feature}
	\begin{comment}
		$<$Don’t really say “System Feature 1.” State the feature name in just a few 
		words.$>$
	\end{comment}
\subsection*{Description and Priority}
	\begin{comment}
		$<$Provide a short description of the feature and indicate whether it is of 
		High, Medium, or Low priority. You could also include specific priority 
		component ratings, such as benefit, penalty, cost, and risk (each rated on a 
		relative scale from a low of 1 to a high of 9).$>$
	\end{comment}
	\textsc{Priority:} High \\
	Application communicates with the user by a command line or generated files. In terms of standardize the actions in a application following requirements have been extracted.
\subsection*{Stimulus/Response Sequences}
	\begin{comment}
		$<$List the sequences of user actions and system responses that stimulate the 
		behavior defined for this feature. These will correspond to the dialog elements 
		associated with use cases.$>$
	\end{comment}
	\stimresp
	{User passes a path to the configuration file.}
	{Application allows user to pass a correct path for configuration file.}
	
	\medskip
	
	\stimresp
	{User passes a path to generated report.}
	{Application allows user to pass a correct path for generated report.}
	
	\medskip
	
	\stimresp
	{User requests for a template of configuration file.}
	{Application generates the template configuration file.}
	
	\medskip
	
	\stimresp
	{User requests for a manual to the program and configuration.}
	{Application prints the manual in current command line window.}
\subsection*{Functional Requirements}
	\begin{comment}
		$<$Itemize the detailed functional requirements associated with this feature.  
		These are the software capabilities that must be present in order for the user 
		to carry out the services provided by the feature, or to execute the use case.  
		Include how the product should respond to anticipated error conditions or 
		invalid inputs. Requirements should be concise, complete, unambiguous, 
		verifiable, and necessary. Use “TBD” as a placeholder to indicate when necessary 
		information is not yet available.$>$
		
		$<$Each requirement should be uniquely identified with a sequence number or a 
		meaningful tag of some kind.$>$
		
		REQ-1:	REQ-2:
	\end{comment}
	
	\begin{functional}{Passing Configuration Path}{Very High}{None}
		\label{fn:communication-feature:passing-configuration-path}
		\term{DESC}{User should be able to pass path to the configuration file with a \emph{--configuration} argument passed to a program.}
		\term{RAT}{In order to pass path to configuration file.}
	\end{functional}

	\begin{functional}{Passing Report Path}{Medium}{None}
		\label{fn:communication-feature:passing-report-path}
		\term{DESC}{User should be able to pass path to the generated \gls{report} directory with a \emph{--report} argument passed to a program.}
		\term{RAT}{In order to pass path to generated report directory.}
	\end{functional} 
	
	\begin{functional}{Generating Template}{Low}{None}
		\label{fn:communication-feature:generating-template}
		\term{DESC}{User should be able to generated template of a configuration file with a \emph{--template} argument passed to a program. If the path after the argument is not specified file should be generated in a current execution directory.}
		\term{RAT}{In order to generate template of configuration.}
	\end{functional} 

	\begin{functional}{Viewing Manual}{High}{None}
		\label{fn:communication-feature:viewing-manual}
		\term{DESC}{User should be able to see a manual of configuration and application with a \emph{--help} argument passed to a program.}
		\term{RAT}{In order to view manual for the application and configuration file.}
	\end{functional} 
	\section{Configuration Feature} \label{s:system-features:configuration-feature}
	\begin{comment}
		$<$Don’t really say “System Feature 1.” State the feature name in just a few 
		words.$>$
	\end{comment}

\subsection*{Description and Priority}
	\begin{comment}
		$<$Provide a short description of the feature and indicate whether it is of 
		High, Medium, or Low priority. You could also include specific priority 
		component ratings, such as benefit, penalty, cost, and risk (each rated on a 
		relative scale from a low of 1 to a high of 9).$>$
	\end{comment}
	\textsc{Priority:}Very High \\
	Requirements from this section relate to the configuration file used in simulation.
\subsection*{Stimulus/Response Sequences}
	\begin{comment}
		$<$List the sequences of user actions and system responses that stimulate the 
		behavior defined for this feature. These will correspond to the dialog elements 
		associated with use cases.$>$
	\end{comment}
	
	\stimresp
	{User changes parameters/fields of configuration file.}
	{System queries user for correct configuration file.}
	
\subsection*{Functional Requirements}
	\begin{comment}
		$<$Itemize the detailed functional requirements associated with this feature.  
		These are the software capabilities that must be present in order for the user 
		to carry out the services provided by the feature, or to execute the use case.  
		Include how the product should respond to anticipated error conditions or 
		invalid inputs. Requirements should be concise, complete, unambiguous, 
		verifiable, and necessary. Use “TBD” as a placeholder to indicate when necessary 
		information is not yet available.$>$
		
		$<$Each requirement should be uniquely identified with a sequence number or a 
		meaningful tag of some kind.$>$
		
		REQ-1:	REQ-2:
	\end{comment}

	\begin{functional}{System Resources}{Very High}{None}
		\label{fn:configuration-feature:system-resources}
		\term{DESC}
		{
			User should be able to specify the amount of a particular \gls{computing node} in a system with following parameters:
			\begin{itemize}
				\item Type of node;
				\item Amount of cores;
				\item Pricing of node in time units.
			\end{itemize}
		}
		\term{RAT}{In order to configure nodes in a simulation.}
	\end{functional}

	\begin{functional}{Machine Operational Cost Per Unit Time}{High}{None}
		\label{fn:configuration-feature:machine-opertional-cost-per-unit-time}
		\term{DESC}
		{
			Application should provide functionality to specify the \gls{MOC}. 
		}
		\term{RAT}{In order to configure \gls{MOC} for simulation.}
	\end{functional}

	\begin{functional}{Queues}{Very High}{\ref{fn:configuration-feature:system-resources}}
		\label{fn:configuration-feature:queues}
		\term{DESC}
		{
			Application should provide functionality to manage properties of a queue in a system:
			\begin{itemize}
				\item type and description;
				\item maximum job execution time;
				\item working time (availability time);
				\item price factor;
				\item amount of reserved hardware resources (nodes of which type).
			\end{itemize} 
		}
		\term{RAT}{In order to configure queues for simulation.}
	\end{functional}

	\begin{functional}{User groups}{Very High}{None}
		\label{fn:configuration-feature:user-groups}
		\term{DESC}
		{
			User should be able to manage properties of a user groups in a system:
			\begin{itemize}
				\item amount of members;
				\item budget range;
				\item maximum number of concurrent jobs per user;
				\item maximum total number of cores simultaneously utilized by user.
			\end{itemize} 
		}
		\term{RAT}{In order to configure user groups for simulation.}
	\end{functional}

	\begin{functional}{Job Distribution}{Medium}{None}
		\label{fn:configuration-feature:job-distribution}
		\term{DESC}
		{
			User should be able to parametrize a exponential probability distribution of a jobs. 
		}
		\term{RAT}{In order to configure job distribution.}
	\end{functional}

	% Validation
		
	\begin{functional}{Validate Nodes}{Very High}{\ref{fn:configuration-feature:system-resources}}
		\label{fn:configuration-feature:valid-nodes}
		\term{DESC}
		{
			Configuration file should contain a valid resource configuration with distinguished \gls{computing node}s. At least once node should be provided. Type of node should be unique, amount of cores integer greater than zero and pricing ratio should greater than zero.  
		}
		\term{RAT}{In order to valid system resources.}
	\end{functional}

	\begin{functional}{Validate \gls{MOC}}{Very High}{\ref{fn:configuration-feature:machine-opertional-cost-per-unit-time}}
		\label{fn:configuration-feature:valid-moc}
		\term{DESC}
		{
			Configuration file should contain a valid value of \gls{MOC} greater than $0$. This field is mandatory.
		}
		\term{RAT}{In order to valid \gls{MOC}.}
	\end{functional}

	\begin{functional}{Validate Queues}{Very High}{\ref{fn:configuration-feature:queues}}
		\label{fn:configuration-feature:valid-queues}
		\term{DESC}
		{
			Configuration file should contain a valid configuration of a queues. At least one queue should be configured.
			Type of queue must be unique. Maximum job execution time, price factor and reservation ratio should a positive number.
			Working time should be a two valued variable specifying time range.
		}
		\term{RAT}{In order to valid queues.}
	\end{functional}

	\begin{functional}{Validate User Groups}{Very High}{\ref{fn:configuration-feature:user-groups}}
		\label{fn:configuration-feature:valid-user-groups}
		\term{DESC}
		{
			Configuration file should contain a valid configuration of user groups. At least one user group should be configured. Amount of members, values of budget range, maximum number of concurrent jobs, maximum resources and maximum total number of utilized cores should be positive value.
		}
		\term{RAT}{In order to valid user groups.}
	\end{functional}

	\begin{functional}{Validate Job Distribution}{Very High}{\ref{fn:configuration-feature:job-distribution}}
		\label{fn:configuration-feature:valid-job-distribution}
		\term{DESC}
		{
			Configuration file should contain a valid parameters of exponential probability distribution.	
		}
		\term{RAT}{In order to valid exponential probability distribution.}
	\end{functional}

	\begin{functional}{Valid Configuration}{Very High}
		{
			\ref{fn:configuration-feature:valid-job-distribution},
			\ref{fn:configuration-feature:valid-moc},
			\ref{fn:configuration-feature:valid-nodes},
			\ref{fn:configuration-feature:valid-queues},
			\ref{fn:configuration-feature:valid-user-groups}
		}
		\label{fn:configuration-feature:valid-configuration}
		\term{DESC}
		{
			Configuration file should contain correct values.
		}
		\term{RAT}{In order to configure job distribution.}
	\end{functional}
		
	\section{Simulation Feature} \label{s:system-features:simulation-feature}
	\begin{comment}
		$<$Don’t really say “System Feature 1.” State the feature name in just a few 
		words.$>$
	\end{comment}

\subsection*{Description and Priority}
	\begin{comment}
		$<$Provide a short description of the feature and indicate whether it is of 
		High, Medium, or Low priority. You could also include specific priority 
		component ratings, such as benefit, penalty, cost, and risk (each rated on a 
		relative scale from a low of 1 to a high of 9).$>$
	\end{comment}
	
	\textsc{Priority: } Very High
	
	Requirements described in this section relate to internal processing actions, that are made during the simulation. Each of the action either uses parameters from the configuration file or is specified within the requirement itself. Requirements for which no external parameters are used the priority is very high due to lack of opportunity to influence the change of behavior.
\subsection*{Stimulus/Response Sequences}
	\begin{comment}
		$<$List the sequences of user actions and system responses that stimulate the 
		behavior defined for this feature. These will correspond to the dialog elements 
		associated with use cases.$>$
	\end{comment}
	
	\stimresp
	{User sets configuration for a simulation}
	{Application await for a correct configuration file.}

	\medskip
	
	\stimresp
	{User starts simulation.}
	{Application starts simulation and uses internal processes based on given parameters in configuration file.}
\subsection*{Functional Requirements}
	\begin{comment}
		$<$Itemize the detailed functional requirements associated with this feature.  
		These are the software capabilities that must be present in order for the user 
		to carry out the services provided by the feature, or to execute the use case.  
		Include how the product should respond to anticipated error conditions or 
		invalid inputs. Requirements should be concise, complete, unambiguous, 
		verifiable, and necessary. Use “TBD” as a placeholder to indicate when necessary 
		information is not yet available.$>$
		
		$<$Each requirement should be uniquely identified with a sequence number or a 
		meaningful tag of some kind.$>$
		
		REQ-1:	REQ-2:
	\end{comment}
	
	\begin{functional}{Job Matching}{Very High}{None}
		\label{fn:simulation-feature:job-matcher}
		\term{DESC}
		{
			Each job in a system should be matched to a different queue based on job's requested resources.
		}
		\term{RAT}{In order to match jobs to queues.}
	\end{functional}

	\begin{functional}{Job Parametrization}{Very High}{None}
		\label{fn:simulation-feature:job-parametrization}
		\term{DESC}
		{
			Job in a system should be parametrized by:
			\begin{itemize}
				\item list of required \gls{computing node} with specified amount of cores per node;
				\item requested time (foreseen execution time, upper boundary).
			\end{itemize}
		}
		\term{RAT}{In order to parametrize job.}
	\end{functional}

	\begin{functional}{Managing of \gls{cutoff}}{Very High}{None}
		\label{fn:simulation-feature:cutoff}
		\term{DESC}
		{
			In system exist \gls{cutoff}. Within that time for each queue no new jobs will start execution if their estimated completion time will exceed \gls{cutoff}.
		}
		\term{RAT}{In order to manage \gls{cutoff} of each queue.}
	\end{functional}

	\begin{functional}{Probability Distribution Model}{Very High}{None}
		\label{fn:simulation-feature:distribution-of-jobs}
		\term{DESC}
		{
			Distribution of any two successive jobs requests and the size of the request (within the user class) shall be modeled by an exponential probability distribution with parameters dependent on the class of user. The following Formula \eqref{eq:exponential-probability-distribution} represents mentioned before distribution:
			\begin{equation} \label{eq:exponential-probability-distribution}
				f(x) = \lambda e^{-\lambda x}
			\end{equation}
			where 
			\begin{itemize}
				\item $\lambda$ is a rate parameter,
				\item $x$ is the time of simulation, $x > 0$
				\item $e$ is the base of the natural logarithm
			\end{itemize}
		}
		\term{RAT}{In order to model probability distribution in a system.}
	\end{functional}

	\begin{functional}{Scheduling Algorithm}{High}{None}
		\label{fn:simulation-feature:scheduling-algorithm}
		\term{DESC}
		{
			Queuing software should grant access under a \gls{FCFS} basis, up to available capacity of the machine.
		}
		\term{RAT}{In order to specify scheduling algorithm.}
	\end{functional}

	\section{Reporting Feature} \label{s:system-features:reporting-feature}
	\begin{comment}
		$<$Don’t really say “System Feature 1.” State the feature name in just a few 
		words.$>$
	\end{comment}

\subsection*{Description and Priority}
	\begin{comment}
		$<$Provide a short description of the feature and indicate whether it is of 
		High, Medium, or Low priority. You could also include specific priority 
		component ratings, such as benefit, penalty, cost, and risk (each rated on a 
		relative scale from a low of 1 to a high of 9).$>$
	\end{comment}
	\textsc{Priority: } Very High \\
	The output of the simulation is in form of report, which should contain all the necessary information for comparison of accounting practices.
	
\subsection*{Stimulus/Response Sequences}
	\begin{comment}
		$<$List the sequences of user actions and system responses that stimulate the 
		behavior defined for this feature. These will correspond to the dialog elements 
		associated with use cases.$>$
	\end{comment}

	\stimresp
	{User requests for queue statistics, for example: \gls{throughput}, average waiting time or \gls{turnaroundRatio}.}
	{Each of queues in the system tracks changes in waiting and executing areas.}
	
	\medskip
	
	\stimresp
	{User requests for machine statistics, for example: working and idle time.}
	{Simulation will track changes in time for each job. This data will be compared at the end of a simulation with a queue statistics.}
	
	\medskip
	
	\stimresp
	{User requests for financial statistics, for example: resulting price paid by the users, operating costs or total income.}
	{Accounting will be made for each user. Operating costs and total income will be calculated.}
\subsection*{Functional Requirements}
	\begin{comment}
		$<$Itemize the detailed functional requirements associated with this feature.  
		These are the software capabilities that must be present in order for the user 
		to carry out the services provided by the feature, or to execute the use case.  
		Include how the product should respond to anticipated error conditions or 
		invalid inputs. Requirements should be concise, complete, unambiguous, 
		verifiable, and necessary. Use “TBD” as a placeholder to indicate when necessary 
		information is not yet available.$>$
		
		$<$Each requirement should be uniquely identified with a sequence number or a 
		meaningful tag of some kind.$>$
		
		REQ-1:	REQ-2:
	\end{comment}

	\begin{functional}{Queue Statistics}{Medium}{None}
		\label{fn:reporting-feature:queue-statistics}
		\term{DESC}
		{
			Simulation report should contain following queue statistics:
			\begin{itemize}
				\item \gls{throughput};
				\item average waiting time;
				\item \gls{turnaroundRatio}
			\end{itemize}
		}
		\term{RAT}{In order to provide queue statistics.}
	\end{functional}

	\begin{functional}{Machine Statistics}{Medium}{None}
		\label{fn:reporting-feature:machine-statistics}
		\term{DESC}
		{
			Simulation report should contain following machine statistics:
			\begin{itemize}
				\item working time;
				\item idle time.
			\end{itemize}
		}
		\term{RAT}{In order to provide machine statistics.}
	\end{functional}

	\begin{functional}{Financial Statistics}{High}{None}
		\label{fn:reporting-feature:financial-statistics}
		\term{DESC}
		{
			Simulation report should contain following financial statistics:
			\begin{itemize}
				\item resulting price paid by the users;
				\item operating costs;
				\item \gls{economicBalance}.
			\end{itemize}
		}
		\term{RAT}{In order to provide financial statistics.}
	\end{functional}

	\begin{functional}{Check Statistics}{Medium}
		{
			\ref{fn:reporting-feature:financial-statistics},
			\ref{fn:reporting-feature:machine-statistics},
			\ref{fn:reporting-feature:queue-statistics}
		}
		\label{fn:reporting-feature:check-statistics}
		\term{DESC}
		{
			Report should contain all the fields described in depended requirements.
		}
		\term{RAT}{In order to provide full report.}
	\end{functional}