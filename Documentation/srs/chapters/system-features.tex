\chapter{System Features} \label{chp:system-features}
	\begin{comment}
		$<$This template illustrates organizing the functional requirements for the 
		product by system features, the major services provided by the product. You may 
		prefer to organize this section by use case, mode of operation, user class, 
		object class, functional hierarchy, or combinations of these, whatever makes the 
		most logical sense for your product.$>$
	\end{comment}
	This chapter is divided into four main sections, which are responsible to contain list of requirements and explain processes within among those requirements. First section \ref{s:system-features:communication-feature} describes all the requirements connected with the user-interface, such us the responses for given action parameters. Next section \ref{s:system-features:configuration-feature} presents the way of configuring the system, containing very specified requirements. Simulation process is described in a \ref{s:system-features:simulation-feature} section. The last section \ref{s:system-features:reporting-feature} standardize the report format and the calculated values.
	
	\input{communication-feature}
	\section{Configuration Feature} \label{s:system-features:configuration-feature}
	\begin{comment}
		$<$Don’t really say “System Feature 1.” State the feature name in just a few 
		words.$>$
	\end{comment}

\subsection*{Description and Priority}
	\begin{comment}
		$<$Provide a short description of the feature and indicate whether it is of 
		High, Medium, or Low priority. You could also include specific priority 
		component ratings, such as benefit, penalty, cost, and risk (each rated on a 
		relative scale from a low of 1 to a high of 9).$>$
	\end{comment}
	\textsc{Priority:}Very High \\
	Requirements from this section relate to the configuration file used in simulation.
\subsection*{Stimulus/Response Sequences}
	\begin{comment}
		$<$List the sequences of user actions and system responses that stimulate the 
		behavior defined for this feature. These will correspond to the dialog elements 
		associated with use cases.$>$
	\end{comment}
	
	\stimresp
	{User changes parameters/fields of configuration file.}
	{System queries user for correct configuration file.}
	
\subsection*{Functional Requirements}
	\begin{comment}
		$<$Itemize the detailed functional requirements associated with this feature.  
		These are the software capabilities that must be present in order for the user 
		to carry out the services provided by the feature, or to execute the use case.  
		Include how the product should respond to anticipated error conditions or 
		invalid inputs. Requirements should be concise, complete, unambiguous, 
		verifiable, and necessary. Use “TBD” as a placeholder to indicate when necessary 
		information is not yet available.$>$
		
		$<$Each requirement should be uniquely identified with a sequence number or a 
		meaningful tag of some kind.$>$
		
		REQ-1:	REQ-2:
	\end{comment}
	
	\begin{functional}{\gls{RNG} Seed}{Low}{None}
		\label{fn:configuration-feature:rng-seed}
		\term{DESC}
		{
			Application should provide functionality to specify the \gls{RNG} seed.
		}
		\term{RAT}{In order to allow repeat simulation.}
	\end{functional}

	\begin{functional}{Job Types}{Very High}{\ref{fn:configuration-feature:system-resources}}
		\label{fn:configuration-feature:job-types}
		\term{DESC}
		{
			User should be able to configure job types in a system, which are related to the resources. Each job type should contain list of tuples with following parameters:
			\begin{itemize}
				\item Type of \gls{computing node} (related to type of \ref{fn:configuration-feature:system-resources});
				\item Probability of occurrence;
				\item Maximum amount of nodes of this type;
				\item Minimum and maximum execution time.
			\end{itemize}
			Additionally each of mentioned before tuples should contain a field \emph{Probability}, which for all the job types sum of their probability should exceed $1$.
		}
		\term{RAT}{In order to allow repeat simulation.}
	\end{functional}

	\begin{functional}{Simulation Time}{High}{None}
		\label{fn:configuration-feature:simulation-time}
		\term{DESC}
		{
			Application should provide functionality to specify the simulation time (beginning and end). 
		}
		\term{RAT}{In order to configure simulation time for simulation.}
	\end{functional}

	\begin{functional}{System Resources}{Very High}{None}
		\label{fn:configuration-feature:system-resources}
		\term{DESC}
		{
			User should be able to specify the amount of a particular \gls{computing node} in a system with following parameters:
			\begin{itemize}
				\item Type of node;
				\item Amount of cores;
				\item Pricing of node in time units.
			\end{itemize}
		}
		\term{RAT}{In order to configure nodes in a simulation.}
	\end{functional}

	\begin{functional}{Machine Operational Cost Per Unit Time}{High}{None}
		\label{fn:configuration-feature:machine-opertional-cost-per-unit-time}
		\term{DESC}
		{
			Application should provide functionality to specify the \gls{MOC}. 
		}
		\term{RAT}{In order to configure \gls{MOC} for simulation.}
	\end{functional}

	\begin{functional}{Queues}{Very High}{\ref{fn:configuration-feature:system-resources}}
		\label{fn:configuration-feature:queues}
		\term{DESC}
		{
			Application should provide functionality to manage properties of a queue in a system:
			\begin{itemize}
				\item type and description;
				\item maximum job execution time;
				\item working time (availability time);
				\item price factor;
				\item amount of reserved hardware resources (nodes of which type).
			\end{itemize} 
		}
		\term{RAT}{In order to configure queues for simulation.}
	\end{functional}

	\begin{functional}{User groups}{Very High}{None}
		\label{fn:configuration-feature:user-groups}
		\term{DESC}
		{
			User should be able to manage properties of a user groups in a system:
			\begin{itemize}
				\item amount of members;
				\item budget range;
				\item maximum number of concurrent jobs per user;
				\item maximum total number of cores simultaneously utilized by user.
			\end{itemize} 
		}
		\term{RAT}{In order to configure user groups for simulation.}
	\end{functional}

	\begin{functional}{Job Distribution}{Medium}{None}
		\label{fn:configuration-feature:job-distribution}
		\term{DESC}
		{
			User should be able to parametrize a exponential probability distribution of a jobs. 
		}
		\term{RAT}{In order to configure job distribution.}
	\end{functional}

	% Validation
		
	\begin{functional}{Valid \gls{RNG} Seed}{Medium}{None}
		\label{fn:configuration-feature:valid-rng-seed}
		\term{DESC}
		{
			Configuration file should contain valid \gls{RNG} seed.
		}
		\term{RAT}{In order to valid \gls{RNG} seed.}
	\end{functional}
	
	\begin{functional}{Valid Job Types}{Very High}{\ref{fn:configuration-feature:valid-nodes}}
		\label{fn:configuration-feature:valid-job-types}
		\term{DESC}
		{
			Sum of probability for job types should be equal $1$. Type of \gls{computing node} should match existing types of nodes in a system. Probability of occurrence should positive. Maximum amount of nodes of this type should be positive. At least one type of job should be specified in a system. Also range of the execution time should be indicated, by two positive values, where upper boundary is greater then lower one.
		}
		\term{RAT}{In order to allow repeat simulation.}
	\end{functional}

	\begin{functional}{Validate Simulation Time}{High}{None}
		\label{fn:configuration-feature:validate-simulation-time}
		\term{DESC}
		{
			The simulation time (beginning and end) should be a positive value related to the dates. 
		}
		\term{RAT}{In order to validate simulation time for simulation.}
	\end{functional}

	\begin{functional}{Validate Nodes}{Very High}{\ref{fn:configuration-feature:system-resources}}
		\label{fn:configuration-feature:valid-nodes}
		\term{DESC}
		{
			Configuration file should contain a valid resource configuration with distinguished \gls{computing node}s. At least once node should be provided. Type of node should be unique, amount of cores integer greater than zero and pricing ratio should greater than zero.  
		}
		\term{RAT}{In order to valid system resources.}
	\end{functional}

	\begin{functional}{Validate \gls{MOC}}{Very High}{\ref{fn:configuration-feature:machine-opertional-cost-per-unit-time}}
		\label{fn:configuration-feature:valid-moc}
		\term{DESC}
		{
			Configuration file should contain a valid value of \gls{MOC} greater than $0$. This field is mandatory.
		}
		\term{RAT}{In order to valid \gls{MOC}.}
	\end{functional}

	\begin{functional}{Validate Queues}{Very High}{\ref{fn:configuration-feature:queues}}
		\label{fn:configuration-feature:valid-queues}
		\term{DESC}
		{
			Configuration file should contain a valid configuration of a queues. At least one queue should be configured.
			Type of queue must be unique. Maximum job execution time, price factor and reservation ratio should a positive number.
			Working time should be a two valued variable specifying time range.
		}
		\term{RAT}{In order to valid queues.}
	\end{functional}

	\begin{functional}{Validate User Groups}{Very High}{\ref{fn:configuration-feature:user-groups}}
		\label{fn:configuration-feature:valid-user-groups}
		\term{DESC}
		{
			Configuration file should contain a valid configuration of user groups. At least one user group should be configured. Amount of members, values of budget range, maximum number of concurrent jobs, maximum resources and maximum total number of utilized cores should be positive value.
		}
		\term{RAT}{In order to valid user groups.}
	\end{functional}

	\begin{functional}{Validate Job Distribution}{Very High}{\ref{fn:configuration-feature:job-distribution}}
		\label{fn:configuration-feature:valid-job-distribution}
		\term{DESC}
		{
			Configuration file should contain a valid parameters of exponential probability distribution.	
		}
		\term{RAT}{In order to valid exponential probability distribution.}
	\end{functional}

	\begin{functional}{Valid Configuration}{Very High}
		{
			\ref{fn:configuration-feature:valid-job-distribution},
			\ref{fn:configuration-feature:valid-moc},
			\ref{fn:configuration-feature:valid-nodes},
			\ref{fn:configuration-feature:valid-queues},
			\ref{fn:configuration-feature:valid-user-groups},
			\ref{fn:configuration-feature:validate-simulation-time},
			\ref{fn:configuration-feature:valid-job-types},
			\ref{fn:configuration-feature:valid-rng-seed}
		}
		\label{fn:configuration-feature:valid-configuration}
		\term{DESC}
		{
			Configuration file should contain correct values.
		}
		\term{RAT}{In order to configure job distribution.}
	\end{functional}
		
	\input{simulation-feature}
	\section{Reporting Feature} \label{s:system-features:reporting-feature}
	\begin{comment}
		$<$Don’t really say “System Feature 1.” State the feature name in just a few 
		words.$>$
	\end{comment}

\subsection{Description and Priority}
	\begin{comment}
		$<$Provide a short description of the feature and indicate whether it is of 
		High, Medium, or Low priority. You could also include specific priority 
		component ratings, such as benefit, penalty, cost, and risk (each rated on a 
		relative scale from a low of 1 to a high of 9).$>$
	\end{comment}

\subsection{Stimulus/Response Sequences}
	\begin{comment}
		$<$List the sequences of user actions and system responses that stimulate the 
		behavior defined for this feature. These will correspond to the dialog elements 
		associated with use cases.$>$
	\end{comment}

\subsection{Functional Requirements}
	\begin{comment}
		$<$Itemize the detailed functional requirements associated with this feature.  
		These are the software capabilities that must be present in order for the user 
		to carry out the services provided by the feature, or to execute the use case.  
		Include how the product should respond to anticipated error conditions or 
		invalid inputs. Requirements should be concise, complete, unambiguous, 
		verifiable, and necessary. Use “TBD” as a placeholder to indicate when necessary 
		information is not yet available.$>$
		
		$<$Each requirement should be uniquely identified with a sequence number or a 
		meaningful tag of some kind.$>$
		
		REQ-1:	REQ-2:
	\end{comment}
