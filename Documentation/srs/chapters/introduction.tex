
\chapter{Introduction}

\section{Purpose}
	\begin{comment}
	$<$Identify the product whose software requirements are specified in this 
	document, including the revision or release number. Describe the scope of the 
	product that is covered by this SRS, particularly if this SRS describes only 
	part of the system or a single subsystem.$>$
	\end{comment}
	The product described in this document is a simulation tool, which will be used by University IT department. The scope of usage of this software is to model the behavior of the \gls{computing-platform}. Accounting practices are to most important and the most desired outcome of the simulation. The requirements described in this document concern parametrization of the simulation, simulator internal processing actions and output of the simulation.
\section{Document Conventions}
	\begin{comment}
		$<$Describe any standards or typographical conventions that were followed when 
		writing this SRS, such as fonts or highlighting that have special significance.  
		For example, state whether priorities  for higher-level requirements are assumed 
		to be inherited by detailed requirements, or whether every requirement statement 
		is to have its own priority.$>$
	\end{comment}
	%DOCUMNET_CONVENCTIONS
	
	All the requirements specified in this document are divided into chapters and sections. Distinguishing: external interfaces, functional features, nonfunctional features and other features. All of them uses \emph{PLanguage} standard. Fragment of a specification can be found in Appendix \ref{apx:planguage-conventions}.
	\begin{comment}		
		\begin{table}[!htbp]
			\centering
			\caption{Impact and likelihood matrix.}
			\label{tab:impact-and-likelihood-matrix}
			\begin{tabular}[width=\textwidth]{cc|c|c|c|c|c|}
				\cline{3-7}
				&  & \multicolumn{5}{c|}{\textit{\textbf{Impact}}} \\ \cline{3-7} 
				&  & \textit{\textbf{NS}} & \textit{\textbf{Minor}} & \textit{\textbf{Major}} & \textit{\textbf{Moderate}} & \textbf{Severe} \\ \hline
				\multicolumn{1}{|c|}{} & \textit{\textbf{Almost Certain}} & \cellcolor[HTML]{FFFE65}\textit{Medium} & \cellcolor[HTML]{F8A102}\textit{High} & \cellcolor[HTML]{FE0000}\textit{Very High} & \cellcolor[HTML]{FE0000}\textit{Very High} & \cellcolor[HTML]{FE0000}\textit{Very High} \\ \cline{2-7} 
				\multicolumn{1}{|c|}{} & \textit{\textbf{Likely}} & \cellcolor[HTML]{FFFE65}\textit{Medium} & \cellcolor[HTML]{F8A102}\textit{High} & \cellcolor[HTML]{F8A102}\textit{High} & \cellcolor[HTML]{FE0000}\textit{Very High} & \cellcolor[HTML]{FE0000}\textit{Very High} \\ \cline{2-7} 
				\multicolumn{1}{|c|}{} & \textit{\textbf{Possible}} & \cellcolor[HTML]{32CB00}\textit{Low} & \cellcolor[HTML]{FFFE65}\textit{Medium} & \cellcolor[HTML]{F8A102}\textit{High} & \cellcolor[HTML]{F8A102}\textit{High} & \cellcolor[HTML]{FE0000}\textit{Very High} \\ \cline{2-7} 
				\multicolumn{1}{|c|}{} & \textit{\textbf{Unlikely}} & \cellcolor[HTML]{32CB00}\textit{Low} & \cellcolor[HTML]{32CB00}\textit{Low} & \cellcolor[HTML]{FFFE65}\textit{Medium} & \cellcolor[HTML]{FFFE65}\textit{Medium} & \cellcolor[HTML]{F8A102}\textit{High} \\ \cline{2-7} 
				\multicolumn{1}{|c|}{\multirow{-5}{*}{\rotatebox[origin=c]{90}{\textit{\textbf{Likelihood}}}}} & \textit{\textbf{Rare}} & \cellcolor[HTML]{32CB00}\textit{Low} & \cellcolor[HTML]{32CB00}\textit{Low} & \cellcolor[HTML]{32CB00}\textit{Low} & \cellcolor[HTML]{32CB00}\textit{Low} & \cellcolor[HTML]{FFFE65}\textit{Medium} \\ \hline
			\end{tabular}
		\end{table}
	
		%FOR TESTERS
		All of requirements have one of four priority levels: \emph{Low}, \emph{Medium}, \emph{High} and \emph{Very High}. Tasks should be completed in order of their priority in descending order, where the lowest priority has the lowest number value. The origin of those values come from the testing priority were we can compare impact of the error and likelihood of its occurrence. This dependency can be clearly seen in Table \ref{tab:priority-and-risk-matrix}. Impact in this case can refer to the weight of a certain task, importance of a task among others. The likelihood can be refer to the probability of occurrence of issue in certain feature. 
		content...
	\end{comment}

	\begin{table}[!htbp]
		\centering
		\caption{Feature to requirement priority table.}
		\label{tab:feature-to-requirement-priority-table}
		\begin{tabular}{cc|c|c|c|c|}
			\cline{3-6}
			&  & \multicolumn{4}{c|}{\textit{\textbf{Features}}} \\ \cline{3-6} 
			&  & \textit{\textbf{Low {[}1{]}}} & \textit{\textbf{Medium {[}6{]}}} & \textit{\textbf{High {[}10{]}}} & \textit{\textbf{Very High {[}20{]}}} \\ \hline
			\multicolumn{1}{|c|}{\multirow{4}{*}{\rotatebox[origin=c]{90}{\textit{\textbf{Functions}}}}} & \textit{\textbf{Low {[}1{]}}} & 1 & 6 & 10 & 20 \\ \cline{2-6} 
			\multicolumn{1}{|c|}{} & \textit{\textbf{Medium {[}2{]}}} & 2 & 12 & 20 & 40 \\ \cline{2-6} 
			\multicolumn{1}{|c|}{} & \textit{\textbf{High {[}4{]}}} & 4 & 24 & 40 & 80 \\ \cline{2-6} 
			\multicolumn{1}{|c|}{} & \textit{\textbf{Very high {[}8{]}}} & 8 & 48 & 80 & 160 \\ \hline
		\end{tabular}
	\end{table}

	%FOR DEVELOPERS
	
	Each of the features introduced in this document has own priority which can be on of the four following: \emph{Low}, \emph{Medium}, \emph{High} and \emph{Very High}. Features contains requirements (functions) which among others within same feature also have a priority level. In order to compare particular requirements among others there should be made a simply calculation presented in Table \ref{tab:feature-to-requirement-priority-table}. Those values are calculated by multiplication values of intersecting columns. 
\section{Intended Audience and Reading Suggestions}
	\begin{comment}
		$<$Describe the different types of reader that the document is intended for, 
		such as developers, project managers, marketing staff, users, testers, and 
		documentation writers. Describe what the rest of this SRS contains and how it is 
		organized. Suggest a sequence for reading the document, beginning with the 
		overview sections and proceeding through the sections that are most pertinent to 
		each reader type.$>$
	\end{comment}
	Software requirements specification document is intended for both development team and stakeholders. Development team contains every person that is responsible for production of the software.
	
	The document is divided into chapters, which is highly recommended to read the whole document from the beginning to the end at least once at the beginning of the introducing to the project. In chapter \ref{chp:overall-description} overall description of a software. It determines type, components and features. This chapter doesn't contain any highly software requirements. Next three chapters (\ref{chp:external-interface-requirements}, \ref{chp:system-features}, \ref{chp:other-nonfunctional-requirements}) contains list of requirements. That requirements are a contract with a customer, which means that all of them should be meet. Chapter \ref{chp:external-interface-requirements} contains informations about the external interfaces, such as user-interfaces, hardware, software and communication. The testing team doesn't have to read all parts of this chapter, because the main goal is to get correct output from the simulation, although aspect of user-friendly software, especially user-interface should be consistent. Managers of the project should be conversant with list of requirements, which implicates into importance of entire document for them. 
\section{Project Scope}
	\begin{comment}
		$<$Provide a short description of the software being specified and its purpose, 
		including relevant benefits, objectives, and goals. Relate the software to 
		corporate goals or business strategies. If a separate vision and scope document 
		is available, refer to it rather than duplicating its contents here.$>$
	\end{comment}
	The purpose of the project is to design and develop simulation tool for supercomputer platform. The goal is to have a tool which by changing different parameters can model different accounting practices of such platform. The benefit is clearly visible, first of all the university could be able to model different accounting practices for different load of jobs/tasks in a system. This kind of approach doesn't have big advantage comparing to real-time introducing changes in accounting practices -- it can be tested in controlled environment. By choosing the best accounting practices, owner of supercomputer can implement competitive prices, which will increase to earnings and utilization of supercomputer.

\section{References}
	\begin{comment}
		$<$List any other documents or Web addresses to which this SRS refers. These may 
		include user interface style guides, contracts, standards, system requirements 
		specifications, use case documents, or a vision and scope document. Provide 
		enough information so that the reader could access a copy of each reference, 
		including title, author, version number, date, and source or location.$>$
	\end{comment}
	%REFERENCES
	
	\begin{enumerate}
		\item \emph{IEEE/ANSI Std. 830-1998 - IEEE Recommended Practice for Software Requirements Specifications}. IEEE Computer Society, 1998.
	\end{enumerate}