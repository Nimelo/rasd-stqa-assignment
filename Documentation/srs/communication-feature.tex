\section{Communication Feature} \label{s:system-features:communication-feature}
	\begin{comment}
		$<$Don’t really say “System Feature 1.” State the feature name in just a few 
		words.$>$
	\end{comment}
\subsection*{Description and Priority}
	\begin{comment}
		$<$Provide a short description of the feature and indicate whether it is of 
		High, Medium, or Low priority. You could also include specific priority 
		component ratings, such as benefit, penalty, cost, and risk (each rated on a 
		relative scale from a low of 1 to a high of 9).$>$
	\end{comment}
	\textsc{Priority:} High \\
	Application communicates with the user by a command line or generated files. In terms of standardize the actions in a application following requirements have been extracted.
\subsection*{Stimulus/Response Sequences}
	\begin{comment}
		$<$List the sequences of user actions and system responses that stimulate the 
		behavior defined for this feature. These will correspond to the dialog elements 
		associated with use cases.$>$
	\end{comment}
	\stimresp
	{User passes a path to the configuration file.}
	{Application allows user to pass a correct path for configuration file.}
	
	\medskip
	
	\stimresp
	{User passes a path to generated report.}
	{Application allows user to pass a correct path for generated report.}
	
	\medskip
	
	\stimresp
	{User requests for a template of configuration file.}
	{Application generates the template configuration file.}
	
	\medskip
	
	\stimresp
	{User requests for a manual to the program and configuration.}
	{Application prints the manual in current command line window.}
\subsection*{Functional Requirements}
	\begin{comment}
		$<$Itemize the detailed functional requirements associated with this feature.  
		These are the software capabilities that must be present in order for the user 
		to carry out the services provided by the feature, or to execute the use case.  
		Include how the product should respond to anticipated error conditions or 
		invalid inputs. Requirements should be concise, complete, unambiguous, 
		verifiable, and necessary. Use “TBD” as a placeholder to indicate when necessary 
		information is not yet available.$>$
		
		$<$Each requirement should be uniquely identified with a sequence number or a 
		meaningful tag of some kind.$>$
		
		REQ-1:	REQ-2:
	\end{comment}
	
	\begin{functional}{Passing Configuration Path}{Very High}{None}
		\label{fn:communication-feature:passing-configuration-path}
		\term{DESC}{User should be able to pass path to the configuration file with a \emph{--configuration} argument passed to a program.}
		\term{RAT}{In order to pass path to configuration file.}
	\end{functional}

	\begin{functional}{Passing Report Path}{Medium}{None}
		\label{fn:communication-feature:passing-report-path}
		\term{DESC}{User should be able to pass path to the generated \gls{report} directory with a \emph{--report} argument passed to a program.}
		\term{RAT}{In order to pass path to generated report directory.}
	\end{functional} 
	
	\begin{functional}{Generating Template}{Low}{None}
		\label{fn:communication-feature:generating-template}
		\term{DESC}{User should be able to generated template of a configuration file with a \emph{--template} argument passed to a program. If the path after the argument is not specified file should be generated in a current execution directory.}
		\term{RAT}{In order to generate template of configuration.}
	\end{functional} 

	\begin{functional}{Viewing Manual}{High}{None}
		\label{fn:communication-feature:viewing-manual}
		\term{DESC}{User should be able to see a manual of configuration and application with a \emph{--help} argument passed to a program.}
		\term{RAT}{In order to view manual for the application and configuration file.}
	\end{functional} 