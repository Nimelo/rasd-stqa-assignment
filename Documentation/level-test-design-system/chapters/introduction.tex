\chapter{Introduction} \label{chp:introduction}
	\begin{comment}
		Introduce the following subordinate sections. This section identifies the issuing organization and the
		details of issuance. It includes required approvals and status (DRAFT/FINAL) of the document. It is
		here that the scope is described and references identified.
	\end{comment}

\section{Document identifier} \label{s:introduction:document-identifier}
	\begin{comment}
		Uniquely identify a version of the document by including information such as the date of issue, the
		issuing organization, the author(s), the approval signatures (possibly electronic), and the status/version
		(e.g., draft, reviewed, corrected, or final). Identifying information may also include the reviewers and
		pertinent managers. This information is commonly put on an early page in the document, such as the
		cover page or the pages immediately following it. Some organizations put this information at the end
		of the document. This information may also be kept in a place other than in the text of the document
		(e.g., in the configuration management system or in the header or footer of the document).
	\end{comment}
\section{Scope} \label{s:introduction:scope}
	\begin{comment}
		Identify the test items (software or system) that are the object of testing, e.g., specific attributes of the
		software, the installation instructions, the user instructions, interfacing hardware, database conversion
		software that is not a part of the operational system) including their version/revision level. Also
		identify any procedures for their transfer from other environments to the test environment.
		Supply references to the test item documentation relevant to an individual level of test, if it exists, such
		as follows:
		⎯ Requirements
		⎯ Design
		⎯ User’s guide
		⎯ Operations guide
		⎯ Installation guide
		Reference any Anomaly Reports relating to the test items.
		Identify any items that are to be specifically excluded from testing.
	\end{comment}
\section{References} \label{s:introduction:references}
	\begin{comment}
		List all of the applicable reference documents. The references are separated into “external” references
		that are imposed external to the project and “internal” references that are imposed from within to the
		project. This may also be at the end of the document.
	\end{comment}
